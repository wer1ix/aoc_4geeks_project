\documentclass{beamer}
\usepackage[utf8]{inputenc}
\usepackage[spanish]{babel}
\usepackage{amsmath}
\usepackage{lmodern}
\usepackage{ragged2e}
\usepackage{enumitem}
\usepackage{xcolor}

% Tema base
\usetheme{Madrid}

% Colores personalizados
\definecolor{violeta}{RGB}{142,68,173} % #8e44ad
\definecolor{rosa}{RGB}{255,105,180}   % #ff69b4
\definecolor{rosapastel}{RGB}{255,182,193}

% Colores Beamer
\setbeamercolor{structure}{fg=violeta} % títulos, secciones
\setbeamercolor{title}{fg=white,bg=violeta}
\setbeamercolor{frametitle}{fg=white,bg=violeta}
\setbeamercolor{itemize item}{fg=rosa}
\setbeamercolor{itemize subitem}{fg=rosapastel}
\setbeamercolor{block title}{fg=white,bg=violeta}
\setbeamercolor{block body}{fg=black,bg=rosapastel!30}

% Opcional: fuente en negrita para títulos
\setbeamerfont{title}{series=\bfseries}
\setbeamerfont{frametitle}{series=\bfseries}

\title{Aplicaciones del Algoritmo Ant Colony Optimization (ACO)}
\author{Judith, Maria Florencia, Monserrat, Mirsha, Ilse}
\date{agosto 19, 2025}

\begin{document}
\frame{\titlepage}


% =================== Slides team
\begin{frame}{Introducción a las metaheurísticas)}

\justifying
\textbf{Del gr.} \textbf{Del gr.} $\epsilon\upsilon\rho\acute{\iota}\sigma\kappa\epsilon\iota\nu$ 
 \textit{heurískein} `hallar', `inventar' y ‒́tico.\\[0.5em]
Técnica de la indagación y del descubrimiento.\\[0.5em]
\textbf{f.} Búsqueda o investigación de documentos o fuentes históricas.\\[0.5em]
\textbf{f.} En algunas ciencias, manera de buscar la solución de un problema mediante métodos no rigurosos, como por tanteo, reglas empíricas, etc.
\end{frame}

\begin{frame}{Historia del Algoritmo ACO}
\begin{columns}
    % Columna de texto
    \begin{column}{0.65\textwidth}
        \justifying
        Fue planteada por primera vez por \textbf{Marco Dorigo} en su tesis doctoral en 1992, inspirado en el comportamiento de las hormigas. 
        Resolvío un problema clásico de optimización combinatoria conocido como el \textbf{Problema del Viajante}, que consiste en encontrar la ruta más corta que permite visitar una serie de ciudades exactamente una vez y volver a la ciudad inicial. 
        
        \medskip
        A lo largo del tiempo, el método ha sido actualizado y perfeccionado, por lo que la versión original ahora se conoce como \textbf{Simple Ant Colony Optimization (SACO)}.
    \end{column}

    % Columna de imagen
    \begin{column}{0.35\textwidth}
        \centering
        \includegraphics[width=\linewidth]{ruta/a/imagen.jpg} % Cambiar ruta/a/imagen.jpg
        \captionof{figure}{Marco Dorigo, 1992}
    \end{column}
\end{columns}
\end{frame}


\begin{frame}{Comportamiento de las hormigas}
\begin{columns}
    % Columna de texto
    \begin{column}{0.65\textwidth}
        \justifying
        Las hormigas al buscar alimento inicialmente exploran aleatoriamente alrededor del nido. 
        Durante el viaje de retorno, la cantidad de feromona que depositan en el suelo puede depender de la calidad y cantidad de alimento hallado.
        
        \medskip
        Dado que las feromonas se evaporan con el tiempo, los caminos más largos pierden su señal antes, mientras que el más corto se refuerza y termina siendo el preferido por la mayoría.
        
        \medskip
        Este proceso colectivo, propio de la \textbf{inteligencia en enjambre}, permite a las hormigas optimizar rutas.
    \end{column}

    % Columna de imagen
    \begin{column}{0.35\textwidth}
        \centering
        \includegraphics[width=\linewidth]{ruta/a/imagen.jpg} % Cambiar por la ruta real
        \captionof{figure}{Hormigas siguiendo rastros de feromonas}
    \end{column}
\end{columns}
\end{frame}


\begin{frame}{Simple Ant Colony Optimization (SACO)}
\justifying
El \textbf{Simple Ant Colony Optimization (SACO)} imita el comportamiento de las colonias reales, utilizando un grupo de ``hormigas artificiales'' para buscar la mejor solución a un problema. 

\medskip
Estas hormigas virtuales dejan ``rastros de feromonas'' para comunicarse entre sí, tal como lo hacen las hormigas reales, y con el tiempo convergen hacia la solución óptima.
\end{frame}

\begin{frame}{Simulación del comportamiento de las hormigas}
\begin{columns}
    % Columna de imagen
    \begin{column}{0.35\textwidth}
        \centering
        \includegraphics[width=\linewidth]{ruta/a/imagen.jpg} % Cambiar por la ruta real
        \captionof{figure}{Esquema de nido y caminos}
    \end{column}

    % Columna de texto
    \begin{column}{0.65\textwidth}
        \justifying
        Para simular este comportamiento establecemos:
        \begin{itemize}
            \item Un nido y una fuente de alimento conectados por dos caminos de longitudes distintas.
            \item Un parámetro que representa la cantidad de feromonas depositadas en cada camino.
            \item Un conjunto de hormigas artificiales en el nido.
            \item Un peso o valor para cada camino, llamado \textbf{heurística} (caminos más cortos reciben un valor mayor).
            \item Una regla de decisión para cada hormiga, donde la probabilidad de elegir un camino depende tanto de la feromona acumulada como de su valor heurístico.
        \end{itemize}
        Se asume que estas hormigas artificiales se comportan como las reales: toman decisiones basadas en la concentración de feromonas, pero sin conocer la longitud de los caminos.
    \end{column}
\end{columns}
\end{frame}

% --- Diapositiva 1: solo texto ---
\begin{frame}{Dinámica de las feromonas}
\justifying
Al inicio, se asigna la misma cantidad de feromonas en ambos caminos, por lo tanto, cada hormiga elegirá aleatoriamente uno de ellos, ya que ambos tienen la misma probabilidad de ser seleccionados.

A medida que las hormigas recorren y depositan feromonas, los caminos que ofrecen soluciones más rápidas o eficientes reciben más refuerzo. Sin embargo, la feromona depositada no permanece indefinidamente: con el tiempo, parte de esta feromona se evapora o desaparece gradualmente.
\end{frame}

% --- Diapositiva 2: con imagen a la izquierda ---
\begin{frame}{Evaporación y exploración}
\begin{columns}
    % Columna imagen
    \begin{column}{0.35\textwidth}
        \centering
        \includegraphics[width=\linewidth]{ruta/a/imagen.jpg} % Cambia ruta a tu imagen
    \end{column}
    % Columna texto
    \begin{column}{0.65\textwidth}
        \justifying
        Este proceso de evaporación evita que las hormigas queden atrapadas en caminos subóptimos y mantiene el equilibrio entre exploración y explotación.  

        Al evaporarse, la concentración de feromonas se reduce, dando oportunidad a que otros caminos menos transitados puedan ser explorados y evaluados.  

        En cada iteración, las hormigas construyen rutas basadas en la concentración de feromonas y la heurística del problema. Este ciclo de construcción de soluciones, actualización y evaporación se repite muchas veces, lo que permite al sistema explorar distintas rutas y reforzar progresivamente las mejores, hasta converger hacia una solución óptima o satisfactoria.
    \end{column}
\end{columns}
\end{frame}















% =================== APLICACIONES
\begin{frame}{Optimización de redes de datos (routing en telecomunicaciones)}
\end{frame}

% === SLIDE 1 ========
\begin{frame}{Optimización de redes de datos (routing en telecomunicaciones)}
\textbf{Símil:} Las hormigas son paquetes de datos buscando el camino más rápido en una red.

\medskip
\textbf{Explicación:} Igual que las hormigas dejan feromonas en caminos cortos, los paquetes “dejan rastro” en rutas rápidas y confiables, reforzando su uso.

\medskip
\textbf{Parámetros clave:}
\begin{itemize}[noitemsep]
    \item $\alpha$: Influencia de feromonas (historia de la red)
    \item $\beta$: Influencia de la heurística (latencia medida)
    \item $\rho$: Tasa de evaporación
    \item $Q$: Intensidad de refuerzo
\end{itemize}

\medskip
\textbf{Fórmulas:}
\[
P_{ij} = \frac{\tau_{ij}^\alpha \cdot \eta_{ij}^\beta}{\sum_{k} \tau_{ik}^\alpha \cdot \eta_{ik}^\beta}
\]
\[
\eta_{ij} = \frac{1}{\text{latencia}_{ij}}, \quad
\tau_{ij} \leftarrow (1-\rho)\tau_{ij} + \frac{Q}{\text{latencia\_ruta}}
\]
\end{frame}

% ======= SLIDE 2 =====
\begin{frame}{Planificación de viajes turísticos}
\textbf{Símil:} Las hormigas son turistas planificando qué ciudades visitar maximizando experiencias y minimizando distancias.

\medskip
\textbf{Explicación sencilla:} El “olor” de las feromonas indica rutas populares y eficientes; la heurística valora atracciones y cercanía.

\medskip
\textbf{Parámetros clave:}
\begin{itemize}[noitemsep]
    \item $\alpha$: Peso de popularidad histórica
    \item $\beta$: Peso de valor de atracciones / distancia
    \item $\rho$: Evaporación para fomentar exploración
\end{itemize}

\medskip
\textbf{Fórmulas:}
\[
\eta_{ij} = \frac{\text{valor\_atractivo}_j}{\text{distancia}_{ij}}
\]
\[
\tau_{ij} \leftarrow (1-\rho)\tau_{ij} + \frac{Q}{\text{utilidad\_ruta}}
\]
\end{frame}

% ===== SLIDE 3 ========
\begin{frame}{Optimización de tráfico vehicular}
\textbf{Símil:} Las hormigas son coches que eligen calles según tiempos de viaje y experiencias previas.

\medskip
\textbf{Explicación:} Las rutas con menos congestión reciben más “feromonas” (preferencia) y se refuerzan, mientras que las congestionadas pierden atractivo.

\medskip
\textbf{Parámetros clave:}
\begin{itemize}[noitemsep]
    \item $\alpha$: Influencia de rutas históricamente rápidas
    \item $\beta$: Influencia de tiempo de viaje real
    \item $\rho$: Evaporación para adaptación a cambios
\end{itemize}

\medskip
\textbf{Fórmulas:}
\[
\eta_{ij} = \frac{1}{\text{tiempo\_viaje}_{ij}}, \quad
\tau_{ij} \leftarrow (1-\rho)\tau_{ij} + \frac{Q}{\text{tiempo\_ruta}}
\]
\end{frame}

% ========== SLIDE 4 =======
\begin{frame}{Optimización de rutas de reparto (logística)}
\textbf{Símil:} Las hormigas son camiones repartiendo paquetes, buscando la ruta más corta para entregar todo.

\medskip
\textbf{Explicación:} Las rutas eficientes se marcan con más feromonas; la heurística favorece caminos cortos.

\medskip
\textbf{Parámetros clave:}
\begin{itemize}[noitemsep]
    \item $\alpha$: Influencia de rutas usadas antes
    \item $\beta$: Influencia de distancias
    \item $\rho$: Evaporación de feromonas
\end{itemize}

\medskip
\textbf{Fórmulas:}
\[
\eta_{ij} = \frac{1}{\text{distancia}_{ij}}
\]
\[
\tau_{ij} \leftarrow (1-\rho)\tau_{ij} + \frac{Q}{\text{longitud\_ruta}}
\]
\end{frame}

% ====== SLIDE 5 ========
\begin{frame}{Asignación de tareas en fábricas}
\textbf{Símil:} Las hormigas son trabajos que deben asignarse a máquinas para minimizar el tiempo total de producción.

\medskip
\textbf{Explicación:} Las combinaciones de tarea-máquina rápidas ganan más feromonas; la heurística valora menor tiempo de proceso.

\medskip
\textbf{Parámetros clave:}
\begin{itemize}[noitemsep]
    \item $\alpha$: Influencia de asignaciones previas exitosas
    \item $\beta$: Influencia de tiempos de proceso
    \item $\rho$: Evaporación para adaptación
\end{itemize}

\medskip
\textbf{Fórmulas:}
\[
\eta_{ij} = \frac{1}{\text{tiempo\_proceso}_{ij}}
\]
\[
\tau_{ij} \leftarrow (1-\rho)\tau_{ij} + \frac{Q}{\text{tiempo\_total}}
\]
\end{frame}

% ======== SLIDE 6 ========
\begin{frame}{Secuenciación de ADN (bioinformática)}
\textbf{Símil:} Las hormigas son intentos de unir fragmentos de ADN en el orden correcto.

\medskip
\textbf{Explicación:} Las feromonas refuerzan empalmes con mayor solapamiento; la heurística premia coincidencias largas.

\medskip
\textbf{Parámetros clave:}
\begin{itemize}[noitemsep]
    \item $\alpha$: Influencia de empalmes anteriores
    \item $\beta$: Influencia de longitud de solapamiento
    \item $\rho$: Evaporación para evitar callejones sin salida
\end{itemize}

\medskip
\textbf{Fórmulas:}
\[
\eta_{ij} = \text{longitud\_solapamiento}(i,j)
\]
\[
\tau_{ij} \leftarrow (1-\rho)\tau_{ij} + \frac{Q}{\text{error\_ensamblaje}}
\]
\end{frame}

\end{document}
